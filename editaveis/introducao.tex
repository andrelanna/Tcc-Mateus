\chapter[Introdução]{Introdução}

A busca pela ``melhor'' configuração ou conjunto de parâmetros para a resposta de um determinado objetivo/problema é uma preocupação recorrente de problemas práticos e teóricos \cite{combinatorialoptimization1998}. Em muitos desses problemas, realizar uma busca exaustiva por todos os estados do espaço de solução afim de encontrar uma condição ideal não é uma alternativa viável do ponto de vista prático devido à enorme quantidade de possibilidades a serem testadas. 

A Ciência da Computação também lida comumente com esses tipos de problemas desde seus primórdios. \citeonline{menabrea1842sketch} relatam tais problemas em suas notas~\cite{menabrea1842sketch}. Um caso particular está descrito na nota D em que as autoras expõem que é preferível escolher um arranjo que tenderá a reduzir o tempo necessário para completar a realização de um cálculo realizado pela 
máquina de calcular de Charles Babbage, do que necessariamente encontrar o arranjo exato da melhor solução possível\cite{menabrea1842sketch}.

A otimização de sistemas é um processo de melhoramento de soluções iterativo e até mesmo interativo tendo em vista sempre uma função-objetivo. Tal processo de resolução de sistemas apresenta-se como uma possibilidade de solução bastante adotada para problemas cujo espaço de solução oferece inúmeras possibilidades. Um tipo especial desses problemas é chamado de problema multi-objetivo, sendo caracterizado por uma função matemática representando diversos objetivos a serem atingidos tendo eles, muitas vezes, propósitos diferentes. Como exemplo a se citar está a busca por uma solução que a) maximize os ganhos de um sistema  com b) o menor esforço possível.  Para esses problemas os métodos de aproximação (também chamados de heurísticas) geram resultados de alta qualidade em um tempo razoável para uso prático e têm sido amplamente empregados como técnicas para otimizar a solução de problemas~\cite{gendreau2005metaheuristics}. Com o desenvolvimento dessa área de pesquisa os métodos de aproximação foram sendo melhor estruturados e conceituados e assim definiram um nível mais generalista e consolidado de heurísticas, as chamadas meta-heurísticas.

Na Engenharia de Software o primeiro trabalho com o uso de meta-heurísticas em um problema de otimização foi na área de testes, especificamente na geração de dados de teste \cite{miller1976automatic}. Em 2001, foi cunhado o termo Engenharia de Software Baseado em Busca (\textit{Search-Based Software Engineering - SBSE}), propondo a reformulação da Engenharia de Software como um problema de busca \cite{harman2001search}. A estruturação proposta provocou a evolução de outras áreas na Engenharia de Software tais como a estimativa de custos e o gerenciamento de projetos. Todavia, os testes continuam dominando o interesse da comunidade de modo que foi estabelecida  uma área própria denominada de Testes de Software Baseados em Busca (\textit{Search-Based Softare Testing - SBST}) \cite{harman2012search}

Um dos principais tópicos pesquisados em SBST é a geração de dados para teste como uma maneira de substituir a escolha empírica dos dados feita pelo testador \cite{mcminn2004search}. A grande maioria das abordagens de geração de dados a partir de meta-heurísticas são de objetivo único, ou seja, estão preocupadas em atingir a cobertura do código que está sendo testado \cite{harman2015achievements}. Entretanto, na realidade, é desejável que um teste consiga atingir outros objetos além de determinados níveis de cobertura como, port exemplo, demandar um esforço reduzido para sua execução em termos de tempo e/ou espaço além da inviolabilidade da realização dos testes\cite{harman2015achievements}.

Isso exposto, a definição de testes tendo como objetivo apenas a maximização da cobertura de código parece não ser realista do ponto de vista dos desenvolvedores e testadores. Em outras palavras, para a definição de um teste deve-se buscar aumentar sua cobertura de código ao mesmo tempo em que ele se mantem exequível, i.e., dentro de um tempo adequado para resposta. Há, portanto, uma necessidade de investigação de outras características de testes e como elas estão relacionadas. Portanto, estabelece-se a seguinte questão de pesquisa para esse trabalho:


\fbox{
\parbox{0.85\textwidth}{\textbf{Questão de pesquisa:} quais são as características de testes que são consideradas pelos desenvolvedores e testadores e como elas se relacionam?}
}

Dentre as características identificadas na literatura duas merecem especial atenção quanto ao esforço computacional demandado. Por um lado o esforço demandado para a geração de dados de teste parece tem uma relação direta com o nível de cobertura atingido. Quanto maior o nível de cobertura desejado, provavelmente mais casos de testes serão demandados e, portanto, mais dados de testes devem ser gerados. Por outro lado, a execução de testes tem, certamente, uma relação direta com o esforço empreendido em tal tarefa. Dados ambas características, tem-se como objetivos maximizar a cobertura de código de um conjunto de testes ao mesmo tempo em que busca-se executá-los no menor tempo possível.

Portanto, o objetivo deste trabalho é compreender o ponto de equilíbrio nas relações existentes na geração/execução de dados de teste a saber: i) o esforço computacional empregado na geração dos dados de teste e a cobertura de código provida por tais dados e ii) o esforço computacional para a execução do teste e o nível de cobertura de código alcançada. Em outras palavras, estima-se que o ponto de equilíbrio entre essas duas relações caracteriza como uma solução de testes com boa cobertura de código e tempo de execução factível. Para tal estudo ambas características serão aplicadas à abordagens que utilzam algoritmos genéticos devido à sua larga utilização nessa área e outros fatores que estão  descritos em detalhes na Seção \ref{propostaestudo}.

Na continuidade do trabalho é necessário o estudo a fundo desse esforço computacional a fim de conseguir definir um aspecto a ser analisado, por exemplo, o tempo de execução. Além disso, a compreensão do comportamento das iterações do algoritmo genético é importante na identificação da sua interferência na cobertura produzida. Com essas definições, será possível descrever as funções de relação e estabelecer a otimização das mesmas. 

O trabalho está disposto em três principais capítulos. O Capítulo \ref{referencialteorico} apresenta o referencial téorico da pesquisa, abordando os conceitos e classificações inerentes à otimização; as meta-heurísticas como métodos probabílisticos de solução para o problemas de solução, caracterizando as principais utilizadas na Engenharia de Software; e a geração de dados de teste baseado em busca, expressando conceitos como cobertura de código e testes multi-objetivos. No Capítulo \ref{propostaestudo} é desenvolvida a proposta de estudo, mostrando a justificativa, os objetivos a as atividades que serão executadas para alcançar a finalidade da pesquisa. O Capítulo \ref{conclusoes}, conclui as ideias apresentadas e mostra as expectativas para o seguimento da pesquisa.
