\chapter[Introdução]{Introdução}

A busca pela ``melhor'' configuração ou conjunto de parâmetros para um
determinado objetivo é uma preocupação recorrente de problemas práticos e
teóricos \cite{combinatorialoptimization1998}. Em muitos desses problemas,
realizar uma busca exaustiva por todos os estados a fim de encontrar uma
condição não é viável devido ao tamanho de possibilidades a serem testadas. 

A ciência da computação, desde seus primórdios, também lida comumente com esses tipos de problemas. Menabrea e Lovelace relatam isso em suas notas. Em especial na nota D, em que expõem que é preferível escolher um arranjo que
tenderá a reduzir o tempo necessário para completar o cálculo, se referindo a
máquina de calcular de Babbage, do que necessariamente o arranjo exato
\cite{menabrea1842sketch}.

A otimização, processo de melhoramento iterativo e até mesmo interativo respeitando uma função objetivo, vem como uma forte solução para essa espécie de problema. Métodos de aproximação (heurísticas) que geram
resultados de alta qualidade, em um tempo razoável para uso prático, vem sendo
adotados como técnicas para otimizar \cite{gendreau2005metaheuristics}. Com o desenvolvimento dessa área de
pesquisa, esses modos de aproximação foram sendo melhor estruturados e
conceituados, criando um nível generalista e consolidado de heurísticas (meta-heurísticas).

Na engenharia de software, o primeiro trabalho com o uso de meta-heurísticas em um problema de 
otimização foi na área de testes, especificadamente na geração de dados de teste \cite{miller1976automatic}. Em 2001, foi cunhado o termo engenharia de software baseado em busca, propondo a reformulação da engenharia de software como um problema de busca \cite{harman2001search}. A estruturação proposta provocou a evolução de outras áreas na engenharia de software, como a estimativa de custos e o gerenciamento de projetos. Todavia, os testes continuam dominando o interesse, criando uma área própria denominada de teste de software baseado em busca \cite{harman2012search}

Em teste de software baseado em busca, o principal problema pesquisado é na geração de dados para teste, substituindo a escolha empírica dos dados pelo testador \cite{mcminn2004search}. A grande maioria das abordagens de geração de dados a partir de meta-heurísticas são de objetivo único, ou seja, estão preocupadas em atingir a cobertura do código que está sendo testado \cite{harman2015achievements}. Entretanto, na realidade, é desejável que um teste consiga proporcionar outras características essenciais, como um esforço reduzido para a execução, seja ele de tempo ou espaço, e a inviolabilidade da realização do teste \cite{harman2015achievements}.

O objetivo da pesquisa é explorar a área de teste multi-objetivo baseado em busca, propondo uma técnica de otimização para maximizar a cobertura de código e minimizar o esforço necessário para a execução. 

 




