\part{Referencial Teórico}

\chapter[Referencial Teórico]{Referencial Teórico}

A busca pela ``melhor'' configuração ou conjunto de parâmetros para um
determinado objetivo é uma preocupação recorrente de problemas práticos e
teóricos \cite{combinatorialoptimization1998}. Em muitos desses problemas,
realizar uma busca exaustiva por todos os estados a fim de encontrar uma
condição não é viável devido ao tamanho de possibilidades a serem testadas. 

Para a solução desses problemas, métodos de aproximação (heurísticas) que geram
resultados de alta qualidade, em um tempo razoável para uso prático, vem sendo
adotados \cite{gendreau2005metaheuristics}. Com o desenvolvimento dessa área de
pesquisa, esses modos de aproximação foram sendo melhor estruturados e
conceituados, criando um nível generalista de heurísticas (meta-heurísticas).

% Essa conceituação abaixo é necessária? Ou com o final do paragrafo anterior é
% possível entender o que são meta-heurísticas? Seria interessante colocar algum
% exemplo? Ou até mesmo falar quando foi primeiro utilizado esse termo de
% heurísticas na solução de problemas de otimização?

% ALPML: certamente! Contudo é necessário detalhar essa explicação e esclarecer
% os vários termos citados. O que é meta-heuristica? O que significa ser de
% nível superior? O que é uma estratégia de orientação? O que é um problema de
% otimização?
Os métodos de busca meta-heurísticos podem ser definidos como metodologias,
modelos, de nível superior que podem ser usados como estratégias de orientação
no projeto de heurísticas subjacentes para resolver problemas específicos de
otimização \cite{talbi2009metaheuristics}.

% É importante voltar tanto assim? Ou quebra muito o contexto? Vi que quebrou um
% pouco o ritmo de leitura, mas achei interessante mencionar

% ALPML: é um bom exemplo para mostrar que a busca por melhores configurações
% vem de muito tempo atrás. Vale mencionar esse exemplo como um dos primórdios
% da busca por boas soluções em problemas em computacao.
A ciência da computação, desde seus primórdios, também lida comumente com
problemas de otimização, como é mostrado por Menabrea e Lovelace em suas notas -
especialmente a nota D, em que relatam que é preferível escolher um arranjo que
tenderá a reduzir o tempo necessário para completar o cálculo, se referindo a
máquina de calcular de Babbage, do que necessariamente o arranjo exato
\cite{menabrea1842sketch}.

Na comunidade acadêmica, as linhas de pesquisa que empregam meta-heurísticas
para a resolução de problemas de otimização são denominadas baseadas em busca,
do inglês \textit{search-based}.

\section{Teste de software baseado em busca}

Na engenharia de software, o primeiro trabalho com o uso de técnicas para
otimização foi na área de testes. Em 1969, King utilizou a execução simbólica
automatizada para checar a consistência entre o programa e suas proposições
\cite{king1969program}. 

Anos depois surgiram os primeiros trabalhos utilizando meta-heurísticas. Como é
o caso do SELECT, sistema para teste e depuração de programas utilizando
execução simbólica aliada com a meta-heurística de subida da encosta
\cite{boyer1975select}.

% É necessário explicar melhor o que esses trabalhos contribuíram?

% ALPML: Sim, apresente mais detalhadamente cada um deles... Tente responder a
% cada uma das perguntas abaixo
% * Qual o contexto em que o trabalho está inserido? Por contexto entenda a área
% de ESW em que ele está inserido (detalhadamente). 
% * Qual o problema que o trabalho lida ou tenta resolver? Como esse problema é
% caracterizado? 
% * Qual a solução proposta no trabalho? Qual técnica de SBSE foi utilizada?
Na mesma época, a utilização de meta-heurísticas se estendeu para a geração de
caso de teste, área que ganhou enorme apreço por pesquisa. O trabalho de geração
automática de dados de ponto flutuante \cite{miller1976automatic} acabou sendo
um marco para os testes de software. Bem como, anos depois, a utilização de
algoritmo genético para geração de dados de teste para a cobertura estrutural
\cite{xanthakis1992application}.

A partir de então, as pesquisas foram combinando entre as meta-heurísticas -
subida de encosto, algoritmos genéticos, colônia de formigas, recozimento
simulado - e os diversos tipos de testes  conhecidos - testes estruturais,
testes funcionais e não funcionais, testes de segurança, testes de robustez,
testes de estresse, testes de mutação, testes de integração e testes de exceção
\cite{harman2009search}.

Além da área de teste, outros ramos da engenharia de software começaram a
utilizar as técnicas de busca, como foi o caso da estimativa de custos e do
gerenciamento de projetos \cite{de2011ten}. O surgimentos de trabalhos fez a
necessidade de melhor estruturar e definir a engenharia de software baseada em
busca.

\subsection{Engenharia de software baseada em busca}

O termo engenharia de software baseada em busca foi pela primeira vez expressado
pelo manifesto, que propôs a reformulação da engenharia de software como um
problema de busca. Para tal reestruturação, é necessário definir uma
representação do problema passível de manipulação simbólica, uma função de
avaliação e um conjunto de operadores de manipulação \cite{harman2001search}.

As definições citadas acimas contribuem para o desenvolvimento da engenharia de
software. Com tais, é possível observar a generalidade que se ganha com a
representação de um problema e com a função de avaliação, e a robustez que os
algoritmos de otimização proporcionam \cite{harman2012search}. 

Além disso, pode ser observada a escalabilidade através do paralelismo que as
meta-heurísticas proporcionam, a reunificação de subáreas da engenharia de
software e a computação direta dos problemas \cite{harman2012search}. Isso,
devido ao software já estar representada de maneira lógica, diferente de outras
engenharias, que precisam da simulação dos seus artefatos. Como por exemplo, a
representação de um material físico, o que necessita de um alto poder
computacional.

Muito se pesquisou desde então sobre a engenharia de software baseada em busca.
Como a área de testes é a de interesse, optou-se por adentrar como os tipos de
testes são explorados.

\subsection{Objetivos pesquisados}

A otimização por busca se encontra difundida em testes, pois é possível ver ela
atuando em todas as dimensões do teste. Seja ele estrutural (caixa branca),
funcional (caixa-preta), uma mescla de ambos (caixa cinza) e não-funcional.

\subsubsection{Teste estrutural}

Essa área, desde os primórdios, foi a que mais recebeu esforços para o
desenvolvimento. Principalmente, visando a geração de casos e dados de teste
para a cobertura de caminhos possíveis na execução. 

A princípio, os casos de testes era individuais, ou seja, as funções de
adaptação era construídas com base no caminho em que seria percorrido.
Ultimamente, se propõe testes com multi-objetivos. 
% Nesse caso acima, tenho citar os artigos que são a referência para o conceito.
% Como, por exemplo, testes com multi-objetivos

% ALPML: exatamente! 
Neste meio tempo, discuti-se também sobre otimização de testes baseada no
conceito à orientação. 


Para cada problema, caracterizá-lo:

- Definição breve e clara do problema

- Propostas de solução(fw, algoritmo, arquitetura...)

- Métodos de otimização empregrado(Alg. Genet., A*, Hill Climbing)

- Resultados alcançados

\subsection{Algoritmos mais utilizados}

\subsubsection{Subida de encosta}

O algoritmo da subida de encosta é um método simples que proporciona resultados rápidos, e por isso é um dos algoritmos mais utilizados. Ele é baseado na busca em profundidade, que é um algoritmo utilizado para travessia em grafos. 

A subida de encosta tem como ponto de partida o espaço de busca e um resultado prévio escolhido arbitrariamente. A cada iteração, os vizinhos desse resultado é examinado. A solução atual é substituída assim que se encontra uma solução melhor. Isso ocorre sucessivamente, até que se não consiga encontra vizinhos melhores.  

\subsubsection{Recozimento simulado}

\subsubsection{Algoritmo genético}








