%\part{Referencial Teórico}

\chapter[Referencial Teórico]{Referencial Teórico}

\section{Problema de otimização}

A otimização é um termo que advém da matemática, a qual em sua essência está preocupada com a obtenção das condições que dão o valor extremo de uma função, ou de várias funções, sob determinadas circunstâncias. A definição formal matemática poder ser dada como: 

\begin{eqnarray}
\label{eqn01}
	 \textsf{Encontrar \textbf{x} que minimize \textbf{f(x)} sujeito a} \nonumber \\
     g_j(x) \leq 0, \qquad j = 1, 2, ..., n_g \\
\label{eqn02}
      h_k(x) = 0, \qquad k = 1, 2, ..., n_h \\
\label{eqn03}
      x_{i}^{L} \leq x_i \leq x_{i}^{U}, \qquad i = 1, 2, ..., n 
\end{eqnarray}

onde $x_{e}$ o vetor das n variáveis de projeto, $f(x)$ é a função objetivo e $g_{j}(x)$ e $h_{k}(x)$ são as restrições de desigualdade e igualdade, respectivamente. Os limites das variáveis são determinados através da equação \ref{eqn03}, onde onde $x_{i}^{L}$ é o limite inferior e $x_{i}^{U}$ é o limite superior da variável $x_{m}$. Nas outras expressões, $n$, $n_{g}$ e $n_{h}$ são o número de variáveis de projeto, número de restrições de desigualdade e igualdade, respectivamente \cite{gandomi2013metaheuristic}.

Na definição, a função objetivo se refere ao que se quer maximizar ou minimizar, o conjunto de variáveis é o que afeta o valor da função objetivo e o conjunto de restrições é não permite que o conjunto de variáveis assuma
determinados valores. Considerando isso, é possível classificar os problemas de otimização de acordo com esses fatores. Tal classificação é possível de observada na figura \ref{figotimizacao01}.

Os problemas típicos de otimização vão desde a minimização de funções algébricas até assuntos mais práticos em projetos de engenharia, como a minimização de custos e a maximização de desempenho. A pergunta que todos esse tem em comum é "Dentre as soluções viáveis, qual é a melhor?". O meio que se busca para respondê-la, após a definição das restrições e da função objetivo, é por meio dos métodos de otimização. Os métodos de otimização que se ajustam a pergunta são os probabilísticos, no caso as heurísticas e meta-heurísticas \cite{gandomi2013metaheuristic}.

\begin{figure}[h]
	\centering
	\label{figotimizacao01}
    \tikzset{
        basic/.style  = {draw, text width=4cm, drop shadow, font=\sffamily, rectangle},
        root/.style   = {basic, rounded corners=2pt, thin, align=center,
                         fill=gray!60},
        onode/.style = {basic, thin, align=center, fill=gray!40,text width=4cm,},
        tnode/.style = {basic, thin, align=left, fill=gray!20, text width=6.5em},
        edge from parent/.style={-, >={latex}, draw=black, edge from parent fork right}
    }

    \begin{tikzpicture} [%
        grow=right,
        anchor=west,
        growth parent anchor=east,
        parent anchor=east,
        level 1/.style={sibling distance=6em},
        level 2/.style={sibling distance=2.5em},
        level distance=0.5cm]
    \node[root] (root) {Classificação dos problemas de otimização}
        child {node[onode] (c1) {Número de variáveis de projeto}
            child {node[tnode] (c11) {Única Variável}}
            child {node[tnode] (c12) {Multivariável}}
        }
        child {node[onode] (c2) {Número de funções objetivo}
            child {node[tnode] (c21) {Único Objetivo}}
            child {node[tnode] (c22) {Multiobjetivo}}
        }
        child {node[onode] (c3) {Presença de restrições}
            child {node[tnode] (c31) {Sem restrições}}
            child {node[tnode] (c32) {Com restrições}}
        }
        child {node[onode] (c4) {Tipo de variáveis de projeto}
            child {node[tnode] (c41) {Discreta}}
            child {node[tnode] (c42) {Contínua}}
            child {node[tnode] (c43) {Mista}}
        }
        child {node[onode] (c5) {Característica das restrições e da função objetivo}
            child {node[tnode] (c51) {Linear}}
            child {node[tnode] (c52) {Não-linear}}
        }
        child {node[onode] (c5) {Natureza das variáveis e dados de entrada}
            child {node[tnode] (c51) {Determinística}}
            child {node[tnode] (c52) {Probabilística}}
        };
    \end{tikzpicture}
    \caption{Classificação dos problemas de otimização \cite{gandomi2013metaheuristic}}
\end{figure}

\section{Meta-heurísticas}

Os métodos de busca meta-heurísticos podem ser definidos como metodologias,
modelos, de nível superior que podem ser usados como estratégias de orientação
no projeto de heurísticas subjacentes para resolver problemas de
otimização \cite{talbi2009metaheuristics}. Ser de nível superior significa que são algoritmos de propósito geral, enquanto as heurísticas são desenhadas e construídas para um problema específico. 

Como dito, elas costumam ser usadas como estratégia de orientação para o projeto de heurísticas subjacentes. Isso expressa que normalmente elas são adaptadas de acordo com as restrições, as variáveis e a função objetivo do problema em questão. 

Em uma meta-heurísticas dois principais fatores são observados: a investigação do espaço de busca (diversificação) e a exploração das melhores soluções encontradas (intensificação). Esses fatores, geralmente estão ligados ao tipo de solução que estão baseadas as meta-heurísticas. As baseadas em solução única, como o recozimento simulado e a busca local, estão ligadas a intensificação e as baseadas na população, como algoritmos genético e a inteligência de enxame, estão relacionadas a diversificação \cite{talbi2009metaheuristics}.

As meta-heurísticas possuem uma grande variabilidade de classificação, que vão além dos fatores mostrados anteriormente. Um outro que pode ser citado é a inspiração, podendo ser baseada nos processos da natureza, a grande maioria, e não baseada em processos naturais. A maneira de decisão do algoritmos classifica as meta-heurísticas em determinísticas e estocásticas, em que regras aleatórias são aplicadas durante a busca. Dado essa diversidade, a Subida de Encosto, o Recozimento Simulado, o Algoritmo Genético e a Inteligência de Enxame ganham destaque, principalmente na área de engenharia de software \cite{khari2017extensive}.

\subsection{Subida de Encosta}

O algoritmo da subida de encosta é um método simples que proporciona resultados rápidos, e por isso é um dos algoritmos mais utilizados. Ele é baseado na busca em profundidade, que é um algoritmo utilizado para travessia em grafos. 

A subida de encosta tem como ponto de partida o espaço de busca e um resultado prévio escolhido arbitrariamente. A cada iteração, os vizinhos desse resultado é examinado. A solução atual é substituída assim que se encontra uma solução melhor. Isso ocorre sucessivamente, até que se não consiga encontra vizinhos melhores.  

\subsection{Recozimento Simulado}

Recozimento simulado é um algoritmo similar a Subida de Encosta, com a diferença que a as movimentos no espaço de busca não são tão limitados. Para a exploração do espaço de busca, ele utiliza um parâmetro de controle denominado temperatura. Ela é a probabilidade de aceitar soluções com menor valor de aptidão, as piores soluções \cite{kirkpatrick1983optimization}.

O método consiste em iniciar com um valor alto de temperatura, e a medida que é aplicado o resfriamento, ou seja, a busca, a temperatura diminui até chegar a zero. Assim como a Subida de Encosto, o Recozimento Simulado considera apenas uma solução por iteração e não realiza qualquer suposição sobre o panorama de soluções. Outro problema que possui igualmente à Subida de Encosto é a possibilidade de ficar preso em um ótimo local quando a temperatura esfria rapidamente.

\subsection{Algoritmo genético}

\subsection{Inteligência de enxame}


\section{Teste multi-objetivo baseado em busca}

Na comunidade acadêmica, as linhas de pesquisa que empregam meta-heurísticas
para a resolução de problemas de otimização são denominadas baseadas em busca, do inglês \textit{search-based}.

Na engenharia de software, o primeiro trabalho com o uso de técnicas para
otimização foi na área de testes. Em 1969, King utilizou a execução simbólica
automatizada para checar a consistência entre o programa e suas proposições
\cite{king1969program}. 

Anos depois surgiram os primeiros trabalhos utilizando meta-heurísticas. Como é
o caso do SELECT, sistema para teste e depuração de programas utilizando
execução simbólica aliada com a meta-heurística de subida da encosta
\cite{boyer1975select}.

% É necessário explicar melhor o que esses trabalhos contribuíram?

% ALPML: Sim, apresente mais detalhadamente cada um deles... Tente responder a
% cada uma das perguntas abaixo
% * Qual o contexto em que o trabalho está inserido? Por contexto entenda a área
% de ESW em que ele está inserido (detalhadamente). 
% * Qual o problema que o trabalho lida ou tenta resolver? Como esse problema é
% caracterizado? 
% * Qual a solução proposta no trabalho? Qual técnica de SBSE foi utilizada?
Na mesma época, a utilização de meta-heurísticas se estendeu para a geração de
caso de teste, área que ganhou enorme apreço por pesquisa. O trabalho de geração
automática de dados de ponto flutuante \cite{miller1976automatic} acabou sendo
um marco para os testes de software. Bem como, anos depois, a utilização de
algoritmo genético para geração de dados de teste para a cobertura estrutural
\cite{xanthakis1992application}.

A partir de então, as pesquisas foram combinando entre as meta-heurísticas -
subida de encosto, algoritmos genéticos, colônia de formigas, recozimento
simulado - e os diversos tipos de testes  conhecidos - testes estruturais,
testes funcionais e não funcionais, testes de segurança, testes de robustez,
testes de estresse, testes de mutação, testes de integração e testes de exceção
\cite{harman2009search}.

\subsection{Engenharia de software baseada em busca}

Além da área de teste, outros ramos da engenharia de software começaram a
utilizar as técnicas de busca, como foi o caso da estimativa de custos e do
gerenciamento de projetos \cite{de2011ten}. O surgimentos de trabalhos fez a
necessidade de melhor estruturar e definir a engenharia de software baseada em
busca.

O termo engenharia de software baseada em busca foi pela primeira vez expressado
pelo manifesto, que propôs a reformulação da engenharia de software como um
problema de busca. Para tal reestruturação, é necessário definir uma
representação do problema passível de manipulação simbólica, uma função de
avaliação e um conjunto de operadores de manipulação \cite{harman2001search}.

As definições citadas acimas contribuem para o desenvolvimento da engenharia de
software. Com tais, é possível observar a generalidade que se ganha com a
representação de um problema e com a função de avaliação, e a robustez que os
algoritmos de otimização proporcionam \cite{harman2012search}. 

Além disso, pode ser observada a escalabilidade através do paralelismo que as
meta-heurísticas proporcionam, a reunificação de subáreas da engenharia de
software e a computação direta dos problemas \cite{harman2012search}. Isso,
devido ao software já estar representada de maneira lógica, diferente de outras
engenharias, que precisam da simulação dos seus artefatos. Como por exemplo, a
representação de um material físico, o que necessita de um alto poder
computacional.

Muito se pesquisou desde então sobre a engenharia de software baseada em busca.
Como a área de testes é a de interesse, optou-se por adentrar como os tipos de
testes são explorados.

\subsection{Objetivos pesquisados}

A otimização por busca se encontra difundida em testes, pois é possível ver ela
atuando em todas as dimensões do teste. Seja ele estrutural (caixa branca),
funcional (caixa-preta), uma mescla de ambos (caixa cinza) e não-funcional.

\subsubsection{Teste estrutural}

Essa área, desde os primórdios, foi a que mais recebeu esforços para o
desenvolvimento. Principalmente, visando a geração de casos e dados de teste
para a cobertura de caminhos possíveis na execução. 

A princípio, os casos de testes era individuais, ou seja, as funções de
adaptação era construídas com base no caminho em que seria percorrido.
Ultimamente, se propõe testes com multi-objetivos. 
% Nesse caso acima, tenho citar os artigos que são a referência para o conceito.
% Como, por exemplo, testes com multi-objetivos

% ALPML: exatamente! 
Neste meio tempo, discuti-se também sobre otimização de testes baseada no
conceito à orientação. 









