\chapter[Apresentação da Proposta de Estudo]{Apresentação da Proposta de Estudo}

A produção de dados para teste de forma automatizada é um dos principais avanços na área de engenharia de software. Isso, porque a necessidade de reproduzir comportamentos para a verificação de um código é fundamental para agregar valor no software. A geração desses dados por meta-heurísticas complementam e contribuem com o desenvolvedor. Todavia, nota-se que nem sempre esses testes gerados são realísticos. Em alguns casos acabam sendo custoso tanto o processo de geração dos dados, quanto a execução deles para medir a cobertura de código. Esses fatores, acabam sendo empecilhos para o crescimento e adoção das meta-heurísticas \cite{harman2015achievements}.

Outro fator observado é que se produz e evolui as meta-heurísticas, mas pouco se estuda e analisa essas técnicas. A geração de dados de teste carece de estudo na avaliação do comportamento das iterações dos algoritmos empregados para a otimização de acordo com os testes produzidos. Algumas métricas que podem ser analisadas, por exemplo, são o tempo de execução e o número de iterações \cite{rodrigues2018using}. 

Os algoritmos genéticos são empregados com bastante frequência para a geração de dados para teste. A sua estruturação em passos (recombinação - mutação - avaliação - seleção) garante uma abrangência e ao mesmo tempo uma especificidade para os dados de teste. As duas primeiras etapas são fundamentais para promoção da variabilidade genética, enquanto que as duas últimas, tomando como base a função objetivo, restringem bem as gerações \cite{rodrigues2018using}. Esses passos são passíveis de otimização e por isso podem ser analisados a fim de que se compreenda o impacto na geração e na execução dos testes.

Dado tais justificativas, abaixo é apresentado os dois principais objetivos da pesquisa, seguido do objetivo geral a ser alcançado. 

\makeatletter
\def\namedlabel#1#2{\begingroup
    #2%
    \def\@currentlabel{#2}%
    \phantomsection\label{#1}\endgroup
}
\begin{description}
\item[\namedlabel{obj1}{Objetivo 1}] Analisar a relação entre cobertura do código e o esforço computacional para geração de dados de teste utilizando algoritmo genético.

A geração de dados para teste está ligada ao processos iterativos da meta-heurística empregada. Os algoritmos genéticos tem a característica de quanto maior a geração, mais custoso é o processo de evolução \cite{pargas1999test}. Por outro lado, essas gerações tendem a proporcionar dados de teste mais específicos. Intuitivamente, percebe-se que é uma relação crescente, mas não se sabe a forma (linear, logarítmica, exponencial). A finalidade é compreender melhor o comportamento dessas duas variáveis e como elas interagem entre si durante a produção dos dados de teste. 
		

\item[\namedlabel{obj2}{Objetivo 2}] Analisar o esforço computacional empregado para execução da cobertura de código a partir dos testes gerados por algoritmo genético.

A depender de como os dados de teste são gerados eles podem provocar um determinado grau de esforço computacional a ser exigido na cobertura de código. Nesse aspecto, também é possível analisar que quanto mais testes são executados, e consequentemente, mais esforço computacional é necessário, maior é a cobertura. Intuitivamente, percebe-se novamente que é uma relação crescente, mas também não se sabe a forma da função. O que se deseja é entender o modo que essas variáveis se relacionam para os algoritmos genéticos.

\end{description}

Com esses objetivos, o que se deseja é encontrar o ponto de equilíbrio, ou seja o ponto ótimo, para ambas análises. Deseja-se estabelecer critérios e métricas para que se consiga analisar os aspectos a serem compreendidos. 

Sendo assim, pode-se caracterizar o estudo de pesquisa como um problema de otimização, no qual se deseja diminuir o esforço computacional empregado para a geração de dados e execução dos teste e maximizar a cobertura de código. Como mostrado, este é um problema multiobjetivo, que se enquadra na área de Testes de Software Baseado em Busca.

\section{Cronograma}

De acordo com a proposta que foi apresentada algumas atividades foram planejadas para dar seguimento na pesquisa. A tabela \ref{cronograma} está divida em meses na vertical e nas atividades, descritas a seguir, na horizontal. 


\begin{description}
\item[\namedlabel{ativ1}{Atividade 1}] Estudar o comportamento das relações referentes ao \ref{obj1} e \ref{obj2}.

\item[\namedlabel{ativ2}{Atividade 2}] Estudar e definir o modo como os esforços computacionais para o \ref{obj1} e \ref{obj2} podem ser medidos.

Como resultado a definição da métrica de esforço computacional.

\item[\namedlabel{ativ3}{Atividade 3}] Propor e analisar uma relação entre o \ref{obj1} e \ref{obj2} que capture os esforços de ambos.

Entender como o comportamentos das duas curvas se equiparam

\item[\namedlabel{ativ4}{Atividade 4}] Escrita dos resultados 


\end{description}

\begin{table}[!htbp]
\centering
\caption{Cronograma das atividades durante os meses de Agosto à Dezembro}
\label{cronograma}
\begin{tabular}{|l|c|c|c|c|c|}
\hline
Mês & \multicolumn{1}{l|}{\multirow{2}{*}{Agosto}} & \multicolumn{1}{l|}{\multirow{2}{*}{Setembro}} & \multicolumn{1}{l|}{\multirow{2}{*}{Outubro}} & \multicolumn{1}{l|}{\multirow{2}{*}{Novembro}} & \multicolumn{1}{l|}{\multirow{2}{*}{Dezembro}} \\ \cline{1-1}
Atividade & \multicolumn{1}{l|}{} & \multicolumn{1}{l|}{} & \multicolumn{1}{l|}{} & \multicolumn{1}{l|}{} & \multicolumn{1}{l|}{} \\ \hline
\ref{ativ1} & X & X &  &  &  \\ \hline
\ref{ativ2} &  & X & X &  &  \\ \hline
\ref{ativ3} &  & X & X & X &  \\ \hline
\ref{ativ4} &  &  &  & X & X \\ \hline
\end{tabular}
\end{table}










