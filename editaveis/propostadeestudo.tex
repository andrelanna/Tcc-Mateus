\chapter[Apresentação da Proposta de Estudo]{Apresentação da Proposta de Estudo}

Introdução ao capítulo (1 parágrafo)
	Fazer um gancho com a contextualização apresentada nos capítulos 1 e 2. 
Mostrar a necessidade de entender a relação entre cobertura de código e esforço de análise de cobertura de código. 
		Resgatar a idéia de que testes atuais não são realísticos: o desenvolvedor olha para outros aspectos do teste além da cobertura. Ex.: esforço para o teste.

Apresentação da proposta (+/- 3 parágrafos)
	Parag. 1: Apresentação em linhas gerais da proposta de estudo:
		Queremos avaliar o ponto em que o esforço para análise de cobertura e a geração de dados de testes que provêem uma cobertura razoável se equivalem. 
		Apresentar brevemente qual a meta-heurística adotada e o motivo.
	
    Parag. 2: Apresentação do objetivo 1 - Geração de dados de teste para uma cobertura razoável.
		Relação cobertura x esforço para geração de dados de teste é, intuitivamente, crescente; -> explicar em detalhes o suficiente para o leitor entender;
		Ainda não sabemos o comportamento dessa relação (linear, logarítmica, exponencial, etc…) -> buscar entendimento sobre esse comportamento e,
		O objetivo para essa relação é sempre de maximizar a cobertura.
	
    Parag. 3: Apresentação do objetivo 2 - Análise do esforço computacional em termos de (??? - tempo, espaço, comparações, etc...) para cobertura de código.
		Relação entre esforço computacional da análise de cobertura e cobertura -> explicar em detalhes o suficiente para o leitor entender;
		Também não se sabe o comportamento dessa relação mas sabe-se intuitivamente, a priori, que é uma relação crescente (linear, logarítmica, exponencial, etc…);
		Objetivo para essa relação é de sempre minimizar o esforço computacional.

	Parag. 4: Conclusão da proposta (1 parágrafo)
		Mostrar que o estudo pretende encontrar o ponto de equilíbrio para os dois objetivos.
		Esse problema pode ser encarado sob a ótica de Testes de Software Baseado em Pesquisa, como um problema de otimização multi-critérios.


\section{Cronograma}


Ativ 1 - Estudar o comportamento das relações referentes aos objetivos 1 e 2, para uma meta-heurística específica (ex. alg. genético, etc…);
Detalhar essa atividade: qual o intuito? o que se espera como resultado?

Ativ 2 - Estudar e definir o modo como os esforços computacionais para os objetivos 1 e 2 podem ser medidos;
Detalhar essa atividade: qual o intuito? O que se espera como resultado? -> definição da métrica de esforço.

Ativ 3 - Propor e analisar uma relação entre os dois objetivos que capture os esforços de ambos os objetivos do estudo.
Entender como o comportamentos das duas curvas se equiparam





